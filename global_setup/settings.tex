%%% --->
% listings settings
%%% <---
\makeatletter
\renewcommand{\ALG@name}{伪代码}
\makeatother

% \renewcommand{\lstlistingname}{Code Snippet}
\renewcommand{\lstlistingname}{代码片段}
\lstdefinestyle{lfonts}{
  basicstyle   = \scriptsize\ttfamily,
  stringstyle  = \color{purple},
  keywordstyle = \color{blue!60!black}\bfseries,
  commentstyle = \color{olive}\scshape,
}
\lstdefinestyle{lnumbers}{
  numbers     = left,
  numberstyle = \tiny,
  numbersep   = 1em,
  firstnumber = 1,
  stepnumber  = 1,
}
\lstdefinestyle{llayout}{
  breaklines       = true,
  tabsize          = 2,
  columns          = spacefixed,
}
\lstdefinestyle{lgeometry}{
  xleftmargin      = 20pt,
  xrightmargin     = 0pt,
  frame            = tb,
  framesep         = \fboxsep,
  framexleftmargin = 20pt,
}
\lstdefinestyle{lgeneral}{
  style = lfonts,
  style = lnumbers,
  style = llayout,
  style = lgeometry,
}
\lstdefinestyle{lmathescape}{
  mathescape = true,
  escapechar = \%,
}
\lstdefinestyle{lcpp}{
  language = {C++},
  style = {lgeneral},
  morekeywords = {constexpr, noexcept, nullptr, override}
}
\lstdefinestyle{lltx}{
  language = {LaTeX},
  style = {lgeneral},
}
\lstdefinestyle{lpy}{
  language = {Python},
  style = {lgeneral},
}
\lstdefinestyle{lsh}{
  language = {sh},
  style = {lgeneral},
}
\lstdefinestyle{ljava}{
  language = {Java},
  style = {lgeneral},
}

%%% --->
% list environment settings
%%% <---
\setlist{nosep}
\setlist[description]{itemsep = 0.5ex}

%%% --->
% figure and TikZ settings
%%% <---
\graphicspath{{./figures/}{./global_figures/}}

\usetikzlibrary{arrows.meta}
\usetikzlibrary{shapes, arrows}
\usetikzlibrary{positioning}
\tikzstyle{decision} = [diamond, draw, fill=blue!20,
    text width=4.5em, text badly centered, node distance=3cm, inner sep=0pt]
\tikzstyle{block} = [rectangle, draw, fill=blue!20,
    text width = 4pc, text centered, rounded corners, minimum height = 2pc]
\tikzstyle{line} = [draw, -latex']
\tikzstyle{cloud} = [draw, ellipse,fill=red!20, node distance=3cm,
    minimum height=2em]
\tikzstyle{autocircle} = [circle, draw = black, fill = gray!30, inner sep = 5pt, minimum size = 0.4cm]
\tikzset{global scale/.style={
    scale=#1,
    every node/.append style={scale=#1}
  }
}
\tikzset{global xscale/.style={
    xscale=#1,
    every node/.append style={xscale=#1}
  }
}
\tikzset{global yslant/.style={
    yslant=#1,
    every node/.append style={yslant=#1}
  }
}

%%% --->
% sectioning settings
%%% <---
\AtBeginDocument{\setcounter{secnumdepth}{0}}
\ctexset{section/format += \raggedright}

%%% --->
% geometry and layout settings
%%% <---
\setlength{\parindent}{0pt}
\addtolength{\parskip}{1ex}
% \geometry{margin = 1in}
\geometry{papersize={14.4cm, 10.8cm}, margin = 0.65in}

%%% --->
% hyperreference settings
%%% <---
\hypersetup{hidelinks}

%%% --->
% theorem environment settings
%%% <---
\newtheorem{note}{提示}
\newtheorem*{note*}{提示}

%%% --->
% sectioning settings
%%% <---
\ctexset{paragraph/beforeskip = {0pt}}

%%% --->
% header & footer settings
%%% <---
\lhead{\relax}
% \lhead{\raisebox{-.5ex}{\textcolor{black}{\large\bfseries\leftmark}}}
\chead{\relax}
\rhead{\relax}
% \rhead{\raisebox{-.5ex}{\makebox[0pt][l]{\quad\large\textcolor{black}{\thepage}}}}
\lfoot{\relax}
\cfoot{\relax}
\rfoot{\relax}
\renewcommand{\headrulewidth}{0pt}
